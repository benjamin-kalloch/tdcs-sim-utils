\section{Setup}
The workflow necessitates two tools for the basic functionality: 1) The segmentation and post-processing of the segmentation images
are realized in the Medical Image Processing And Visualization tool $($MIPAV$)$ \cite{mcauliffe2001medical} with the Java Imaging
and Science Toolkit $($JIST$)$ \cite{lucas2010java} and the CBSTools. Their setup will not be further discussed here as it has
been described before\footnote{https://figshare.com/s/f2e47b41e974ab67039b}\cite{kalloch2018semi}. 2) The OpenFOAM software
package is used for setting up and simulating the test cases, i.e. the tdcs simulations. We will shortly describe where to obtain
this package, which other software is required and how to set everything up.

\subsection{System requirements}
\paragraph{Hardware}
The workflow was tested on a workstation equipped with a 4-core Intel\textsuperscript{\textregistered}~Core\texttrademark~i7 6700
processor clocked at 3,4 GHz with 32 GB of main memory and a 512 GB SSD drive. Major parts of the pipeline including the volume
meshing as well as the solver application are not parallelized. For that reason, a processor with a high single-thread performance
is preferred over a multi-processor system. In addition, both during segmentation as well as during the actual simulation intermediate
results are written and re-read from the hard drive. Therefore, fast storage such as an SSD drive positively influences the
processing speed. Storage requirements for a complete run of the pipeline with all intermediate results amount to approximately
30 GB which can be reduced if intermediate results are discarded after successful completion of the respective processing step.

\paragraph{Software}
The software pipeline is confirmed to work on Ubuntu 16.04 LTS and Ubuntu 18.04 LTS, but more recent versions of Ubuntu should
work as well.

\subsection{Meshing - CGAL, GMSH}
We provide a custom application implementing the API of CGAL version 4.13.1. The version distributed with Ubuntu 19.04 is lacking
an important fix. More recent versions of Ubuntu distribute a more recent version of CGAL, which may be incompatible with our meshing tool.
For this reason, your must either compile CGAL 4.13.1 from the source or build a Docker container using the Docker file we provide (in the
``docker" directory of \emph{meshheadmodel} sub-module). This Docker file creates a Docker
container which is capable of compiling and running the meshing tool. To build the Docker container execute
\newline
\lstinline[basicstyle=\ttfamily]{docker build -t  cgal_and_gmsh_compiled_with_tbb_ubuntu1904 .}
\newline
Before doing so, make sure to place the (unpacked) source code directories of GMSH 4.3 (from: \url{http://gmsh.info/src/gmsh-4.3.0-source.tgz})
and CGAL 4.13.1 (from: \url{https://github.com/CGAL/cgal/releases/download/releases%2FCGAL-4.13.1/CGAL-4.13.1.tar.xz}) into the ``docker"
directory and name them \emph{gmsh} and \emph{cgal} respectively. They are required to compile the meshing tool and will be copied into the
container and built subsequently, so you may delete them after building the container.\par
You may use the \emph{docker\_compile.sh} script to recompile the meshing tool when you've made changes to the source code.\par
When compiled in the container, the tool also must be launched from inside the container. You can do this by calling the \emph{docker\_run.sh}
script which invokes the \emph{runMeshingTool.sh} script from inside the container. When using docker to run the tool consider that all
paths within runMeshingtTool.sh (both input files and output files) must be specified according to the Docker environment (i.e. the shared folder)
NOT according to the host machine. The best practice is to use the predefined input \& output directories and to use paths relative to the
MeshHeadModel executable.\par
Note that for both docker scripts you may have to adjust the variables \$BASE\_DIR and \$CONTAINER\_NAME if you use a path/name other then
predefined in the file.

\subsection{Electrode Modeling \& Positioning - Blender}
The modeling and positioning of the electrodes is implemented as a plugin for the 3D modeling software Blender. The plugin is
confirmed to work up to version 2.79b which you can install using the apt package manager or download it from 
\url{https://www.blender.org/download/Blender2.79/blender-2.79b-linux-glibc219-x86\_64.tar.bz2/}.

\subsection{Test case setup \& Postprocessing - OpenFOAM}
OpenFOAM version 6 provides the framework for the setup and execution of the simulation test cases. Since we extend the package
with a custom solver application you will need the source code of OpenFOAM and compile it. You can download the source code of
OpenFOAM 6 here: \url{https://github.com/OpenFOAM/OpenFOAM-6}. For the the post-processing functionality, especially ParaView
version 5.4, please download the "Third Party" software package as well \url{https://github.com/OpenFOAM/ThirdParty-6}. Since we
provide custom plugins for ParaView you need the source code of ParaView and compile it.

\subsection{Processing of diffusion-weighted data}
Our pipeline allows the use of diffusion-weighted (DWI) data for conductivity tensor estimation. To process DWI data, we use
MRTrix3 which can be obtained from \url{http://www.mrtrix.org/download/}. For MRTrix to work properly you will also need
the FMRIB Software Library (FSL) \url{https://fsl.fmrib.ox.ac.uk/fsldownloads\_registration} and the Advanced Normalization Tools
which can be downloaded here \url{https://stnava.github.io/ANTs/}.\\
In addition, we provide plugins for ParaView v.5.4 that implement the computation of the conductivity tensors. The version of
ParaView that is distributed with OpenFOAM is sufficient for that purpose. For some manual tasks functionality of MIPAV is required.


